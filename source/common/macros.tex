\newcommand{\markedPivot}[1]{\boxed{#1}}
\newcommand{\IR}{\mathbb{R}}
\newcommand{\IC}{\mathbb{C}}
\renewcommand{\P}{\mathcal{P}}
\renewcommand{\Im}{\operatorname{Im}}
\newcommand{\RREF}{\operatorname{RREF}}
\newcommand{\vspan}{\operatorname{span}}
\newcommand{\setList}[1]{\left\{#1\right\}}
\newcommand{\setBuilder}[2]{\left\{#1\,\middle|\,#2\right\}}
\newcommand{\unknown}{\,{\color{gray}?}\,}
\newcommand{\drawtruss}[2][1]{%
\begin{tikzpicture}[scale=#1, every node/.style={scale=#1}]
\draw (0,0) node[left,magenta]{C} -- 
      (1,1.71) node[left,magenta]{A} -- 
      (2,0) node[above,magenta]{D} -- cycle;
\draw (2,0) -- 
      (3,1.71) node[right,magenta]{B} -- 
      (1,1.71) -- cycle;
\draw (3,1.71) -- (4,0) node[right,magenta]{E} -- (2,0) -- cycle;
\draw[blue] (0,0) -- (0.25,-0.425) -- (-0.25,-0.425) -- cycle;
\draw[blue] (4,0) -- (4.25,-0.425) -- (3.75,-0.425) -- cycle;
\draw[thick,red,->] (2,0) -- (2,-0.75);
#2
\end{tikzpicture}
}
\newcommand{\trussNormalForces}{%
\draw [thick, blue,->] (0,0) -- (0.5,0.5);
\draw [thick, blue,->] (4,0) -- (3.5,0.5);
}
\newcommand{\trussCompletion}{%
\trussNormalForces
\draw [thick, magenta,<->] (0.4,0.684) -- (0.6,1.026);
\draw [thick, magenta,<->] (3.4,1.026) -- (3.6,0.684);
\draw [thick, magenta,<->] (1.8,1.71) -- (2.2,1.71);
\draw [thick, magenta,->] (1.6,0.684) -- (1.5,0.855);
\draw [thick, magenta,<-] (1.5,0.855) -- (1.4,1.026);
\draw [thick, magenta,->] (2.4,0.684) -- (2.5,0.855);
\draw [thick, magenta,<-] (2.5,0.855) -- (2.6,1.026);
}
\newcommand{\trussCForces}{%
\draw [thick, blue,->] (0,0) -- (0.5,0.5);
\draw [thick, magenta,->] (0,0) -- (0.4,0.684);
\draw [thick, magenta,->] (0,0) -- (0.5,0);
}
\newcommand{\trussStrutVariables}{%
\node[above] at (2,1.71) {\(x_1\)};
\node[left] at (0.5,0.866) {\(x_2\)};
\node[left] at (1.5,0.866) {\(x_3\)};
\node[right] at (2.5,0.866) {\(x_4\)};
\node[right] at (3.5,0.866) {\(x_5\)};
\node[below] at (1,0) {\(x_6\)};
\node[below] at (3,0) {\(x_7\)};
}
\newcommand{\N}{\mathbb N}
\newcommand{\Z}{\mathbb Z}
\newcommand{\Q}{\mathbb Q}
\newcommand{\R}{\mathbb R}
\DeclareMathOperator{\arcsec}{arcsec}
\DeclareMathOperator{\arccot}{arccot}
\DeclareMathOperator{\arccsc}{arccsc}
\newcommand{\tuple}[1]{\left\langle#1\right\rangle}

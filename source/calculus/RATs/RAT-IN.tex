\documentclass{article}
\usepackage[utf8]{inputenc}

\usepackage[paperwidth=8.5in, paperheight=11in, top=1in, bottom=.5in, left=.5in, right=.5in]{geometry}
\usepackage{fancyhdr, graphicx,tikz,amsmath,multicol,paracol}
\usepackage[inline]{enumitem}


\pagestyle{fancy}
\lhead{\large{\textbf{Module 4: Integration - Readiness Assurance Test}}}
\chead{}
\rhead{}
\lfoot{}
\cfoot{}
%\rfoot{\thepage/\pageref{LastPage} }
\setlength{\headheight}{14pt} %added in bc warning

%%% The answers will be aligned to IF-AT Form B012, questions 41-50. 

%OLD - B021
% 1 A
% 2 D
% 3 D
% 4 C
% 5 A
% 6 A
% 7 C
% 8 C
% 9 A
% 10 B

% 1 D
% 2 B
% 3 C
% 4 A
% 5 D
% 6 A
% 7 D
% 8 A
% 9 C
% 10 B


\begin{document}


\begin{enumerate}


%derivative practice; polynomial, root, negative exponent

\item Let $f(x)=x^3-4x^2+5$. Find $f'(x)$.

  \begin{enumerate}
  
 
  \item $f'(x) = \frac{1}{4}x^4-\frac{4}{3}x^3+5x$ %anti-derivative
  \item $f'(x) = \frac{1}{4}x^4-\frac{4}{3}x^3+5$ 
  \item $f'(x) = 3x^2-8x +5$ %left constant in
  \item $f'(x) = 3x^2-8x$ %correct
  \end{enumerate}



\item Let $g(x)=\sqrt[3]{x^2}$. Find $g'(x)$.

  \begin{enumerate}
  
  
  \item $g'(x) = \dfrac{3}{2}\sqrt{x}$ % derivative of x^3/2 instead
  \item $g'(x) = \dfrac{2}{3\sqrt[3]{x}}$ %correct
  \item $g'(x) = \dfrac{3}{2\sqrt[3]{x}}$ %wrong coefficient 
  \item $g'(x) = \dfrac{2}{3}\sqrt[3]{x}$
  
  
  \end{enumerate}

\item Let $h(x)=\dfrac{3}{x^4}$. Find $h'(x)$.

  \begin{enumerate}
  \item $h'(x) = 12x^3$ %did not change sign of exponent
  \item $h'(x) = -\dfrac{12}{x^3}$ %added 1 to exponent
  \item $h'(x) = -\dfrac{12}{x^5}$ %correct
  \item $h'(x) = \dfrac{3}{4x^3}$ %took derivative of bottom only
  
  \end{enumerate}


% visual of intersection of graphs

\item Let $f(x)=x^2$ and $g(x)=x+6$. At what points do $f(x)$ and $g(x)$ intersect? 

  \begin{enumerate} %include answer choices around correct one
  
  \item $(-2,4)$ and $(3,9)$ %correct
  \item $(2,4)$ and $(-3,9)$
  \item $(4,-2)$ and $(9,3)$
  \item $(4,2)$ and $(9,-3)$
  \end{enumerate}

%area of rectangle in coordinate plane

\item A rectangle is placed in the coordinate plane with its vertices at the points $(2,3)$, $(7,3)$, $(7,7)$ and $(2,7)$. What is the area of this rectangle?

  \begin{enumerate} %include answer choices around correct one
  
  \item 14 square units
  \item 16 square units
  \item 18 square units
  \item 20 square units %correct
  \end{enumerate}


% area of shapes

\item A circle is placed in the coordinate plane such that its center is at $(3,1)$ and the point $(3,4)$ lies on the boundary. What is the area of this circle?

    \begin{enumerate} %include answer choices around correct one
  \item $9\pi$ square units %correct
  \item $16\pi$ square units %used r=4
  \item $3\pi$ square units %didn't square r
  \item $6\pi$ square units %circumference
  \end{enumerate}

\pagebreak
% sigma notation

\item Evaluate the sum:
  \[\sum_{i=1}^{4} (3i-2)  \]
  
    \begin{enumerate} %include answer choices around correct one
  \item 16
  \item 18
  \item 20
  \item 22 %correct
 
  \end{enumerate}

%chain rule (IF YOU GET TO U-SUB IN YOUR CLASS! You may want to edit based on your wording in class!)

\item Let's consider the composite function $f(x)=\sin(x^3+1)$. If you were asked to find $f'(x)$, using the chain rule would be a good approach. What would the inside function be in this case?

  \begin{enumerate}
  
  \item $x^3+1$ % correct
  \item $\sin x$
  \item $x^3$
  \item $\sin(x^3+1)$
  \end{enumerate}

%area of trapezoid in coordinate plane

\item A trapezoid is placed in the coordinate plane with its vertices at the points $(-1,0)$, $(1,0)$, $(1,6)$ and $(-1,8)$. What is the area of this trapezoid?

  \begin{enumerate} %include answer choices around correct one
  
  \item 10 square units
  \item 12 square units
  \item 14 square units %correct
  \item 16 square units
  
  \end{enumerate}

\end{enumerate}


\end{document}

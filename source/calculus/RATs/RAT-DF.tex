\documentclass{article}
\usepackage[utf8]{inputenc}

\usepackage[paperwidth=8.5in, paperheight=11in, top=1in, bottom=.5in, left=.5in, right=.5in]{geometry}
\usepackage{fancyhdr, graphicx,tikz,amsmath,multicol,paracol}
\usepackage[inline]{enumitem}


\pagestyle{fancy}
\lhead{\large{\textbf{Module 2: Differentiation - Readiness Assurance Test}}}
\chead{}
\rhead{}
\lfoot{}
\cfoot{}
%\rfoot{\thepage/\pageref{LastPage} }
\setlength{\headheight}{14pt} %added in bc warning

%%% The answers are aligned to IF-AT Form B012questions 21-30. 


% 1 C
% 2 C
% 3 D
% 4 B
% 5 B
% 6 A
% 7 D
% 8 C
% 9 A
% 10 A


\begin{document}


\begin{enumerate}


% plugging expressions into functions

\item Let $f(x)=x^2-2x+1$. Find $f(3+h)-f(3)$.

  \begin{enumerate}
  
  \item $h^2-2h$ %simplified (3+h)^2 to 9+h^2
   \item $h^2+8h-10$ %did not distribute negative signs
   \item $h^2+4h$ %correct
   \item $h^2-2h-1$ %simplified f(3+h)-f(3) to f(h) (canceled 3's)
  \end{enumerate}

% average rate of change

\item A function $f(x)$ is represented in the table below. What is the average rate of change over the interval $4 \leq x \leq 8$?

\begin{center}
\begin{tabular}{c|rrrrr}
$x$ & 2 & 4 & 6 & 8 & 10 \\ \hline
$f(x)$ & 1 & -5 & -3 & 7 & 25 \\
\end{tabular}

\end{center}

  % add in sequential choices to match answer key
 
  \begin{enumerate}
  \item $1$
  \item $2$
  \item $3$ % correct
  \item $4$
 
  \end{enumerate}
 
  
%equation of line given point and slope

\item Find an equation of the line through the point $(9,-2)$ having slope $\dfrac{2}{3}$. 

  \begin{enumerate}
  \item $y=\dfrac{2}{3}x-2$ %uses y-coordinate as y-intercept
  \item $y=\dfrac{2}{3}x-4$
  \item $y=\dfrac{2}{3}x-6$
  \item $y=\dfrac{2}{3}x-8$ %correct
  \end{enumerate}

%3 alternative: equation of line given point and slope using point-slope

% \item Find an equation of the line through the point $(9,-2)$ having slope $\dfrac{2}{3}$. 

%   \begin{enumerate}
%   \item $y=\dfrac{2}{3}(x+9) -2 $ 
%   \item $y=\dfrac{2}{3}(x+9) +2 $ 
%     \item $y=\dfrac{2}{3}(x-9) -2 $  %correct
%   \item $y=\dfrac{2}{3}(x-9) +2 $ 
%   \end{enumerate}


%composition of functions
\item Let $f(x) = e^x$ and $g(x)=2x-4$. Find $f(g(2))$.

% add in sequential choices to match answer key

  \begin{enumerate}
  \item $0$
  \item $1$  %correct
  \item $2$
  \item $3$
  \end{enumerate}
  

\item The function $f(g(x))=\sqrt{\sin x}$ was created by composing two functions, $f(x)
$ and $g(x)$. What could $f(x)$ and $g(x)$ be?

  \begin{enumerate}

  \item $f(x)=\sqrt{\sin}$, $g(x) =\sqrt{x}$
   \item $f(x)=\sqrt{x}$, $g(x) = \sin x$  %correct 
   \item $f(x)=\sqrt{x}$, $g(x) =\sqrt{\sin}$
   \item $f(x)=\sin x$, $g(x) = \sqrt{x}$
 
  \end{enumerate}



%exponent rules

\item Which expression is equivalent to $\sqrt[4]{x^7}$?

  \begin{enumerate}
   \item $x^{7/4}$  %correct
     \item $x^{4/7}$
     \item $\dfrac{x^4}{x^7}$
  \item $\dfrac{x^7}{x^4}$

 
  \end{enumerate}

\pagebreak
 
\item Which expression is equivalent to $\dfrac{1}{2x^5}$?

  \begin{enumerate}
  \item $2x^{5}$  
  \item $2x^-{5}$ 
  \item $\dfrac{1}{2}x^{5}$  
  \item $\dfrac{1}{2}x^{-5}$ %correct
  \end{enumerate}  

\item In which quadrant of the unit circle do you find positive sine values and negative cosine values?

  \begin{enumerate}
  \item Quadrant IV
  \item Quadrant III
  \item Quadrant II %correct
 
   \item Quadrant I
  \end{enumerate}

%8 alternative: values of sine and cosine
% \item Which of the following are correct values of the sine and cosine functions?

%   \begin{enumerate}
  
%   \item $\sin(\pi)=0, \cos(0)=0$
%   \item $\cos(\pi)=1, \sin(0)=0$
%   \item $\sin(\frac{\pi}{2})=0, \cos(0)=1$
%   \item $\sin(\frac{\pi}{2})=1, \cos(0)=1$ %correct
%   \end{enumerate}


\end{enumerate}


\end{document}

\documentclass{article}


\usepackage[utf8]{inputenc}
\usepackage[paperwidth=8.5in, paperheight=11in, top=1in, bottom=.5in, left=.5in, right=.5in]{geometry}
\usepackage{fancyhdr, graphicx,tikz}
\usepackage{tasks}




\pagestyle{fancy}
\chead{\large{\textbf{Module 7 (CO): Coordinates and Vectors  - Readiness Assurance Test}}}
%\lhead{\large{\textbf{AI - Readiness Assurance Test}}}
%\chead{}
\rhead{}
\lfoot{}
\cfoot{}
%\rfoot{\thepage/\pageref{LastPage} }

\begin{document}

  %  \begin{tasks}[label=(\alph*),label-width=2em](4)
  %   \task Opt A
  %   \task Opt B
  %   \task Opt C
  %   \task Opt D    
  % \end{tasks} \vfill


\begin{enumerate}
% Problem 1: Function notation
\item Let \(y = \sqrt{12-3x}\). Determine the \(y\)-value if \(x=1\).
   \begin{tasks}[label=(\alph*),label-width=2em](4)
    \task \(1\) 
    \task \(2\)
    \task \(3\) %% correct
    \task Undefined
    \task None of these
  \end{tasks} 

% Problem 2: Inverse function notation
\item Let \(y = \sqrt{12-3x}\). Determine the \(x\)-value if \(y=6\).
   \begin{tasks}[label=(\alph*),label-width=2em](4)
    \task \(-8\)  %% correct
    \task \(-2\) 
    \task Undefined
    \task \(3\sqrt{2}\)
    \task None of these 
  \end{tasks} 
  \vspace{1in}
  
% Problem 3, 4, 5: Unit Circle
\item Determine the exact value of \( \sin\left(\frac{7\pi}{6}\right) \).
   \begin{tasks}[label=(\alph*),label-width=2em](5)
    \task \(-\displaystyle\frac{\sqrt{3}}{2}\)
    \task \(-\displaystyle\frac{1}{2}\) %% correct
    \task \(1\)
    \task \(\displaystyle\frac{\sqrt{2}}{2}\)
    \task None of these
  \end{tasks} 

\item Determine the exact value of \( \cos\left(\frac{4\pi}{3}\right) \).
   \begin{tasks}[label=(\alph*),label-width=2em](5)
    \task \(-\displaystyle\frac{\sqrt{3}}{2}\)
    \task \(-\displaystyle\frac{1}{2}\) %% correct 
    \task \(1\)
    \task \(\displaystyle\frac{\sqrt{2}}{2}\)
    \task None of these  
  \end{tasks} 

\item Determine the exact value of \( \tan\left(\frac{5\pi}{4}\right) \).
   \begin{tasks}[label=(\alph*),label-width=2em](5)
    \task \(-\displaystyle\frac{\sqrt{3}}{2}\)
    \task \(-\displaystyle\frac{1}{2}\) 
    \task \(1\) %% correct
    \task \(\displaystyle\frac{\sqrt{2}}{2}\)
    \task None of these  
  \end{tasks} \vfill

\newpage 
% Arctangent
\item Evaluate \( \arctan(\sqrt{3}) \).
   \begin{tasks}[label=(\alph*),label-width=2em](4)
    \task \(\displaystyle \frac{4\pi}{3}\)
    \task \(\displaystyle \frac{\pi}{3}\)  %% correct
    \task \(\displaystyle \frac{2\pi}{3}\)
    \task \(\displaystyle \frac{\pi}{3}+k\pi\), where \(k \in \mathbb{Z}\)
    \task All of the above
  \end{tasks} \vfill


% Equation of a circle
\item Determine the equation of a circle centered at \((2,0)\) with radius \(r=5\).
   \begin{tasks}[label=(\alph*),label-width=2em](3)
    \task \((x-2)^2 + y^2 = 25\) %% correct 
    \task \((x+2)^2 + y^2 = 5\)
    \task \(x^2-2 + y^2 = 25\)
    \task \(x^2 + (y-2)^2 = 25\)
    \task None of these
  \end{tasks} \vfill

% Derivative evaluation
\item Find \( f'(1) \) for the function \(f(x) = (x-2)(x-3) \).
   \begin{tasks}[label=(\alph*),label-width=2em](5)
    \task \(1\)
    \task \(2\)
    \task \(-3\) %% correct
    \task \(-1\)    
    \task \(0\)
  \end{tasks} \vfill

% Tangent line equation
\item Determine the equation of the tangent line to \(f(x) = (x-2)(x-3) \) when \(x=1\). 
   \begin{tasks}[label=(\alph*),label-width=2em](4)
    \task \(y= -3x\)
    \task \(y= -3(x-1)+2\) %% correct
    \task \(y= -3x+2\)
    \task \(y= -3(x-1)+5\) 
    \task \(y=(1-2)(1-3)\)
  \end{tasks} \vfill


% ArcLength Integral
\item Consider the curve \(y=-3x^3\) defined on \([0,1]\). Find an integral that computes the arc length of this curve.
   \begin{tasks}[label=(\alph*),label-width=2em](2)
    \task \(\displaystyle \int_0^1 \sqrt{1+(-3x^3)^2} \; dx \)
    \task \(\displaystyle \int_0^1 (-3x^3) \; dx \)
    \task \(\displaystyle \int_0^1 2\pi (-3x^3) \sqrt{1+(-9x^2)^2} \; dx \) 
    \task \(\displaystyle \int_0^1 \sqrt{1+(-9x^2)^2} \; dx \)  %% correct
    \task None of these
  \end{tasks} \vfill

% 
    
\end{enumerate}


\end{document}
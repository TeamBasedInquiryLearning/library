%%%%%%%%%%%%%%%%%%%%%%%%%%%%%%%%%%%%%%%%%%%%%%%%%%%%%%%%%%%%%%%
%
% Welcome to Overleaf --- just edit your LaTeX on the left,
% and we'll compile it for you on the right. If you open the
% 'Share' menu, you can invite other users to edit at the same
% time. See www.overleaf.com/learn for more info. Enjoy!
%
%%%%%%%%%%%%%%%%%%%%%%%%%%%%%%%%%%%%%%%%%%%%%%%%%%%%%%%%%%%%%%%
\documentclass{article}
\usepackage[utf8]{inputenc}

\usepackage[paperwidth=8.5in, paperheight=11in, top=1in, bottom=.5in, left=.5in, right=.5in]{geometry}
\usepackage{fancyhdr, graphicx,tikz,amsmath,multicol,paracol}
\usepackage[inline]{enumitem}


\pagestyle{fancy}
\lhead{}
\chead{\large{\textbf{Introduction to TBIL --- Readiness Assurance Test}}}
\rhead{}
\lfoot{}
\cfoot{}
%\rfoot{\thepage/\pageref{LastPage} }
\setlength{\headheight}{14pt} %added in bc warning

%%% The answers are aligned to IF-AT Form D012

%D012
%A
%A
%D
%C
%A
%D
%B
%B
%A
%A


\begin{document}

\begin{enumerate}[itemsep=0.5in]


%A
    \item Which of the following is \textbf{NOT} one of the three categories of TBIL activities?

                  \begin{enumerate}[label=\Alph*)]
                      \item Conceptual Constructor %Correct
                      \item Scaffolded Exploration
                      \item Fluency Builder
                      \item Flexible Extension

                  \end{enumerate}



%A
   \item Which of the following describes the 4-S structure of TBIL activities?

                  \begin{enumerate}[label=\Alph*)]
                      \item Same problem, Significant problem, Specific choice, Simultaneous Response %Correct
                      \item Same problem, Specific Problem, Significant Choice, Simultaneous Response
                      \item Significant problem, Specific choice, Simultaneous Response, Scaffolded problem
                      \item Same problem, Scaffolded problem, Specific choice, Simultaneous Response 
                  \end{enumerate}

%D
    \item Which of the following is the correct chronological sequence of TBIL instruction?

            \begin{enumerate}[label=\Alph*)]
                  
                  \item Assessment, Application Activities, Readiness Assurance Process
                  \item Readiness Assurance Process, Assessment, Application Activities
                  \item Application Activities, Readiness Assurance Process, Assessment
                  \item Readiness Assurance Process, Application Activities, Assessment %Correct
              \end{enumerate}

%C
    \item TBIL is a specific form of which of the following pedagogies?


          \begin{enumerate}[label=\Alph*)]
              
            \item Flipped Learning
            \item Process Oriented Guided Inquiry Learning
          \item Inquiry-Based Learning %Correct
              \item Peer Led Team Learning
              

          \end{enumerate}


% Section 3, Pages 5--6
%A
     \item Which of the following challenges was TBIL \textbf{NOT} designed to address?

                  \begin{enumerate}[label=\Alph*)]
                      \item Improving students' teamwork and communication skills %Correct
                      \item A lack of structure inherent to inquiry-oriented learning
                      \item Obtaining student collaboration and buy-in
                      \item Differing preparation levels of students
                      
                  \end{enumerate}
                  
\newpage

%D
\item What is the most important purpose of the Readiness Assurance Process?
                  \begin{enumerate}[label=\Alph*)]
\item Holds students individually accountable for coming to class prepared
\item Creates a social learning environment where students can compare their understanding of course concepts
\item Delays feedback so students are forced to review and reflect on the right answers for the tRAT
\item Turns initial individual preparation into true readiness to begin problem solving %Correct
\end{enumerate}

%B
\item Why is it important to have teams simultaneously report decisions?
                  \begin{enumerate}[label=\Alph*)]
\item It gives a sense of ``team spirit'' to students who aren’t prepared.
\item It generates vigorous within-a-team and between teams discussion.%Correct
\item It forces ``slower'' teams to be ready.
\item It doesn’t reward the ``smarter'' teams.
\end{enumerate}


%B
\item 
It is important to create TBL teams that are…
\begin{enumerate}[label=\Alph*)]
\item Selected by students to minimize initial student resistance
\item Large, diverse, and instructor selected %Correct
\item Small enough that everyone must pull his or her weight
\item Grouped with similar abilities 
\end{enumerate}



\end{enumerate}


\end{document}